\documentclass{xdyy-usermanual}

\xdyymanualsetup{
  info = {
    author            = {夏康玮},
    title             = { \cls{xdyy-usermanual} 文类},
    email             = {kangweixia_xdyy@163.com},
    date              = {2022-02-01},
    version           = {0.0.1},
    github-repository = {https://github.com/xkwxdyy/xdyy-usermanual},
    gitee-repository  = {https://gitee.com/xkwxdyy/xdyy-usermanual},
  }
}

\begin{document}
\maketitle
\tableofcontents


\section{文类简介}
本文类基于 \cls{l3doc} 文类编写,增加一些自定义命令环境,并且用 \LaTeX3 设计了文档基本信息接口。

主要用于编写宏包或文类的用户手册。


\section{命令环境说明}


\subsection{ \cls{l3doc} 中的常用命令环境 }

\begin{function}{\meta}
  \begin{syntax}
    |\meta{content}|
  \end{syntax}
  产生 \meta{content} 带尖括号的效果
\end{function}

\begin{function}{\marg}
  \begin{syntax}
    |\marg{content}|
  \end{syntax}
  “必选参数”,产生 \marg{content} 的效果
\end{function}

\begin{function}{\oarg}
  \begin{syntax}
    |\oarg{content}|
  \end{syntax}
  “可选参数”,产生 \oarg{content} 的效果
\end{function}



\begin{function}{\tn}
  \begin{syntax}
    |\tn{content}|
  \end{syntax}
  “\LaTeXe{}的命令”,产生 \tn{content} 的效果
\end{function}

\begin{function}{\cs}
  \begin{syntax}
    |\cs{content}|
  \end{syntax}
  “\LaTeX3 的函数”,里面可以嵌套 \tn{meta} 命令。产生 \cs{content} 的效果
  \begin{hexample}
    \cs{\meta{name}:\meta{args}}
  \end{hexample}
\end{function}

\begin{function}{\cmd}
  \begin{syntax}
    |\cmd{content}|
  \end{syntax}
  “命令”,产生 \cmd{content} 的效果
\end{function}
\begin{function}{\env}
  \begin{syntax}
    |\env{content}|
  \end{syntax}
  “环境”,产生 \env{content} 的效果
\end{function}
\begin{function}{\pkg}
  \begin{syntax}
    |\pkg{content}|
  \end{syntax}
  “宏包”,产生 \pkg{content} 的效果
\end{function}
\begin{function}{\cls}
  \begin{syntax}
    |\cls{content}|
  \end{syntax}
  “文类”,产生 \cls{content} 的效果
\end{function}
\begin{function}{\file}
  \begin{syntax}
    |\file{content}|
  \end{syntax}
  “文件”,产生 \file{content} 的效果
\end{function}


\subsection{ \cls{l3doc} 外定义的命令}

\begin{function}{\xdyymanualsetup}
  \begin{syntax}
    |\xdyymanualsetup| = |{ info = |\marg{ \kvopt{key}{val}}| }|
  \end{syntax}
  键值列表,目前只有用户手册的个人信息接口
  \begin{latexcode}
    \xdyymanualsetup{
      info = {
        author            = {夏康玮},
        title             = {用户手册},
        email             = {kangweixia_xdyy@163.com},
        date              = {2022-02-01},
        version           = {0.0.1},
        github-repository = {https://github.com/xkwxdyy/foo},
        gitee-repository  = {https://gitee.com/xkwxdyy/foo},
      }
    }
  \end{latexcode}
\end{function}

\begin{function}{\init}
  \begin{syntax}
    |\init{content}|
  \end{syntax}
  变量或键值的初始值,例子:
\end{function}
\begin{vexample}
    \begin{function}{label-vsep, label-hsep}
      \begin{syntax}
        |label-vsep| = \marg{dimension}  \init{0pt}
        |label-hsep| = \marg{dimension}  \init{0pt}
      \end{syntax}
      \cmd{label} 的垂直和水平的额外偏移量
    \end{function}
\end{vexample}

\begin{function}{\kvopt}
  \begin{syntax}
    |\kvopt{left content}{right content}|
  \end{syntax}
  为键值定义的命令,一般这么用:
  \begin{hexample}
    \kvopt{key}{val}
  \end{hexample}
\end{function}


\subsection{ \cls{l3doc} 外定义的环境}


\subsubsection{ 注释类 }

\begin{function}{remark}
  \begin{syntax}
    |\begin{remark}|
      |  ...|
    |\end{remark}|
  \end{syntax}
  常见的 \env{remark} 环境,效果如下:
  \begin{vexample}
    \begin{remark}
      请注意,2022你会暴富!
    \end{remark}
  \end{vexample}
\end{function}



\subsubsection{ 代码显示 }

\begin{function}{latexcode}
  \begin{syntax}
    |\begin{latexcode}|
      |  ...|
    |\end{latexcode}|
  \end{syntax}
  用于输出 \LaTeX{} 的代码
\end{function}

\begin{function}{bashcode}
  \begin{syntax}
    |\begin{bashcode}|
      |  ...|
    |\end{bashcode}|
  \end{syntax}
  用于输出 \cmd{cmd} 或终端 的代码
\end{function}



\subsubsection{代码例子一起显示}
来自死扣的博客 \href{https://sikouhjw.github.io/2020/04/29/2020-04-29-Make-an-example-environment-with-minted}{《用 minted 宏包制作 example 环境》}

\begin{remark}
  目前下面三个环境的 \cmd{body} 相对于 \env{...example} (\kvopt{...}{h, v, float})环境要有两次缩进才能得到正常不缩进的效果。
\end{remark}
如要得到
\begin{latexcode}
    \begin{enumerate}
      \item item1
      \item item2
    \end{enumerate}
\end{latexcode}
要使用
\begin{latexcode}
    \begin{...example}
        \begin{enumerate}
          \item item1
          \item item2
        \end{enumerate}
    \end{...example}
\end{latexcode}
才行

\begin{function}{hexample}
  \begin{syntax}
    |\begin{hexample}|
      |    ...|
    |\end{hexample}|
  \end{syntax}
  代码和例子水平
\end{function}
\begin{hexample}
    \begin{equation}\label{eq:1}
      E = mc^2
    \end{equation}
    公式~(\ref{eq:1}) 是著名的质能方程。
\end{hexample}

\begin{function}{vexample}
  \begin{syntax}
    |\begin{vexample}|
      |    ...|
    |\end{vexample}|
  \end{syntax}
  代码和例子竖直
\end{function}
\begin{vexample}
    \begin{equation}\label{eq:2}
      E = mc^2
    \end{equation}
    公式~(\ref{eq:2}) 是著名的质能方程。
\end{vexample}


\begin{function}{floatexample}
  \begin{syntax}
    |\begin{floatexample}|
      |    ...|
    |\end{floatexample}|
  \end{syntax}
  浮动体专用
\end{function}
\begin{floatexample}
    \begin{figure}[htbp]
      \centering
      \includegraphics[width=5cm]{example-image.pdf}
      \caption{content}
    \end{figure}
    \end{floatexample}
\end{document}
